\chapter{Introduction}
This report is made in regard to the course \textsf{\courseno\ \coursename} at DTU (Technical University of Denmark). In this second assignment we are to try different regression and classification techniques on our data set. For regression we will try to predict one feature from the information we have from the rest of the features, while in classification we continue with the obvious classification problem which is to determine which digit class an observation belongs to. 

We want to examine the best parameters for each technique based on the error rate for the fitted models and compare them to determine which of them perform better. Then finally we do some comparison between different techniques to see their performance differences.


All the work carried out in this report are done by Tange, M. K. and Sørensen, P. K. and the code can be found on github\footnote{https://github.com/mktange/IntroMachineLearning}. Do note that the code have only been used to generate plots and illustrations for the report and some playing around. There have been done no effort into readability or re-usability of the code.

\section{Data}
The same data as from assignment one have been used, a feature representation of the MNIST dataset. The representation consist of 272 features calculated from vertical, horizontal and radial histograms together with two profiles; in-out, out-in. Refer the first assignments for the full description. 