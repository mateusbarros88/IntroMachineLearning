\chapter{Introduction}
This report is made in regard to the course \textsf{\courseno\ \coursename} at DTU (Technical University of Denmark). In this first assignment we are to pick a data set which we can analyse and extract some feature from as well as try to visualize the things we learn about it.

We have chosen a data set containing a lot of hand-written digits which have been digitalized such that each digit can represented by a vector of pixel values. From this we have performed some manipulation and extracted certain features we found to be wide and conclusive enough to be able to do the machine learning we would like to do. \\

This report goes over which attributes we extract from the raw pixel values, as well as how these can help us tell the different classes in the data set apart. We look at the standard deviation for each class along with the correlation between the classes to get a better picture on how they interact.

Upon doing PCA (Principal Component Analysis) we can get a new look at the data, and determine how many PCs (Principal Components) we need to have a good chance of distinguishing each class from the others. Moreover we can also see how different combinations of PCs are good at telling different classes apart.\\

All the work carried out in this report are done by Tange, M. K. and Sørensen, P. K. and the code can be found on github\footnote{https://github.com/mktange/IntroMachineLearning}. Do note that the code have only been used to generate plots and illustrations for the report and some playing around. There have been done no effort into readability or re-usability of the code.